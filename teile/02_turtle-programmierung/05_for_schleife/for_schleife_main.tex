% !TEX root = ../../../main.tex

\toggletrue{image}
\toggletrue{imagehover}
\chapterimage{loop}
\chapterimagetitle{\uppercase{Loop}}
\chapterimageurl{https://xkcd.com/1411/}
\chapterimagehover{Ugh, today's kids are forgetting the old-fashioned art of absentmindedly reading the same half-page of a book over and over and then letting your attention wander and picking up another book.}


\chapter{Schleifen}
\label{ch:schleifen}

Mit einer \textbf{Schleife} kann Python die Wiederholung von Befehlen direkt übernehmen, ohne dass wir die Befehle mehrmals eintippen müssen. Die Lernziele für dieses Kapitel sind:\\

\lernziel{\autoref{ch:schleifen}, \nameref{ch:schleifen}}{
\begin{minipage}{\linewidth}
$\square$ \hspace{0.1cm} Sie verwenden eine \lstinline[language={python3}]{for}-Schleife, um Befehle wiederholt auszuführen.\\
$\square$ \hspace{0.1cm} Sie setzen die \lstinline[language={python3}]{range}-Funktion im Schleifenkopf ein.\\
$\square$ \hspace{0.1cm} Sie verwenden \lstinline[language={python3}]{import}-Aliasing für eine noch kompaktere Notation.
\end{minipage}
}

\section{Quadrat ultra kompakt}
\label{sec:quadrat-ultra-kompakt}

Die bisherigen \say{Turtle-Programme} waren meist sehr repetitiv. Schon im Programm für das Quadrat der Seitenlänge $100$ (\autoref{lst:quadrat}) wiederholen sich die Funktionsaufrufe \lstinline[language={python3}]{fd(100)} und \lstinline[language={python3}]{lt(90)} jeweils viermal. Wir können nun ein Programm in Python erstellen (siehe \autoref{lst:for-loop-quad}), um diese beiden Funktionsaufrufe durch Python viermal wiederholen zu lassen.

\begin{lstlisting}[language={python3}, caption={Die Zeilen \num{4} und \num{5} werden jeweils viermal ausgeführt (\graybgtexttt{quadrat\_loop.py}).}, label={lst:for-loop-quad}, lineskip={4.0pt}]
import turtle

for _ in [0, 1, 2, 3]:
	turtle.fd(100)
	turtle.lt(90)
turtle.done()

\end{lstlisting}

\vspace{-0.7cm}

\begin{hinweis}
Das Zeichen \lstinline[language={python3}]{	} soll verdeutlichen, dass hier eine Einrückung mit der \textbf{Tabulatortaste} (kurz: \textbf{Tab}) erfolgen \textbf{muss}.
\end{hinweis}

\vspace{-0.25cm}

Die \lstinline[language={python3}]{for}-Schleife besteht aus zwei Teilen (Schleifenkopf und Schleifenkörper):

\begin{itemize}
\item \textbf{Schleifenkopf}: \lstinline[language={python3}]{for} und \lstinline[language={python3}]{in} sind Python-Schlüsselwörter und wir müssen diese exakt so notieren. Dazwischen notieren wir einen \textbf{Underscore} (Unterstrich/Bodenstrich). Danach notieren wir in den \textbf{eckigen Klammern} so viele Werte, wie Wiederholungen durchgeführt werden sollen. Den Schleifenkopf beenden wir mit einem \textbf{Doppelpunkt}.
\end{itemize}

\begin{example}
In \autoref{lst:for-loop-quad} befindet sich der Schleifenkopf in Zeile \num{3}. In den eckigen Klammern stehen vier Werte, es werden also vier Wiederholungen durchgeführt, wenn wir das Programm ausführen.
\end{example}

\begin{itemize}
\item \textbf{Schleifenkörper}: In diesem Schleifenteil notieren wir die Befehle, die wiederholt ausgeführt werden sollen. Diese Befehle müssen wir gleichmässig \textbf{einrücken}. Alle eingerückten Befehle müssen den gleichen Abstand zum \say{Rand} haben. Die Einrückung dient \textbf{nicht} der optischen \say{Verschönerung}. Nur durch die Einrückung (engl. indentation) weiss Python, welche Befehle wiederholt werden müssen. Die eingerückten Befehle bilden den Schleifenkörper.
\end{itemize}

\begin{example}
In \autoref{lst:for-loop-quad} bilden die Zeilen \num{4} und \num{5} den Schleifenkörper, da diese Zeilen eingerückt sind. Die Zeile \num{6} gehört nicht mehr dazu, da sie nicht mehr eingerückt ist.
\end{example}

\cleancoderegel{\autoref{ch:schleifen}, \nameref{ch:schleifen}}{
\begin{cleancode}[Leerzeichen 3]
In den \textbf{eckigen Klammern} wird \textbf{nach} jedem \textbf{Komma} ein \textbf{Leerzeichen} eingefügt.
\end{cleancode}
}

\section{Typische Fehlerquellen}

Bei der Programmierung ist auf die folgenden typischen Fehler zu achten:

\begin{itemize}
\item Schlüsselwörter falsch geschrieben.
\item Doppelpunkt am Ende des Schleifenkopfs vergessen.
\item Schleifenkörper ist falsch eingerückt.
\end{itemize}
\autoref{lst:for-loop-quad-errors} zeigt ein fehlerhaftes Programm mit \textbf{drei Fehlern}. In Zeile \num{3} ist das Schlüsselwort \lstinline[language={python3}]{IN} falsch geschrieben und der Doppelpunkt am Ende der Zeile fehlt. In Zeile \num{5} ist die Einrückung falsch. Der Befehl muss um ein Leerzeichen nach rechts eingerückt werden (oder noch besser: Tabulatortaste verwenden).

\begin{lstlisting}[language={python3}, caption={Achtung, dies ist ein \textbf{fehlerhaftes} Programm.}, label={lst:for-loop-quad-errors}]
import turtle

for _ IN [0, 1, 2, 3]
	turtle.fd(100)
   turtle.lt(90)
turtle.done()

\end{lstlisting}

\cleancoderegel{\autoref{ch:schleifen}, \nameref{ch:schleifen}}{
\begin{cleancode}[Einrückung 1]
Die Einrückung (engl. indentation) des \textbf{Schleifenkörpers} erfolgt mit der Tabulatortaste.
\end{cleancode}
}

\cleancoderegel{\autoref{ch:schleifen}, \nameref{ch:schleifen}}{
\begin{cleancode}[Leerzeichen 4]
Nach dem Schlüsselwort \lstinline[language={python3}]{in} (im Schleifenkopf) fügen wir \textbf{ein Leerzeichen} ein.
\end{cleancode}
}

\section{\lstinline[language={python3}]{range}-Funktion}
\label{sec:range-Funktion}

Wir können den Schleifenkopf etwas kompakter gestalten, indem wir die eckigen Klammern samt Inhalt durch einen \lstinline[language={python3}]{range}-Funktionsaufruf ersetzen. \autoref{lst:for-loop-quad-range} zeigt die Verwendung der \textbf{eingebauten Funktion}. Wir können uns den Funktionsaufruf so vorstellen, als würden wir in den eckigen Klammern die Zahlen $0$, $1$, $2$ und $3$ notieren (siehe \autoref{lst:for-loop-quad-range-comparison}).

\begin{figure}[htb]
\centering
\begin{minipage}{0.45\textwidth}
\centering
\begin{lstlisting}[language={python3}, caption={Eingangsbeispiel}, label={lst:for-loop-quad-range-comparison}]
import turtle

for _ in [0, 1, 2, 3]:
	turtle.fd(100)
	turtle.lt(90)
turtle.done()
\end{lstlisting}
\end{minipage}
\hfill
\begin{minipage}{0.45\textwidth}
\centering
\begin{lstlisting}[language={python3}, caption={Erste Verbesserung.}, label={lst:for-loop-quad-range}]
import turtle

for _ in range(4):
	turtle.fd(100)
	turtle.lt(90)
turtle.done()
\end{lstlisting}
\end{minipage}

\end{figure}

\begin{important}
Der Funktionsaufruf von \lstinline[language={python3}]{range} erfolgt \textbf{ohne} ein davor notierter Modulname. Wir müssen auch kein Modul importieren, um den Funktionsaufruf einzusetzen. \lstinline[language={python3}]{range} ist eine \textbf{eingebaute Funktion (eng. built-in function)}. Diese Funktionsaufrufe sind \textbf{immer} und \textbf{überall} (ohne Import) durchführbar.
\end{important}

\section{\lstinline[language={python3}]{import}-Aliasing}
\label{sec:import-aliasing}

Unsere Programme werden noch kompakter, wenn wir bei \lstinline[language={python3}]{import}-Anweisungen einen Alias (dt. Parallelbezeichnung) verwenden. Dadurch müssen wir nicht jedes Mal den vollständigen Modulnamen verwenden, sondern den von uns gewählten Alias.

\begin{example}
\autoref{lst:import-alias} zeigt ein Beispiel mit einem \lstinline[language={python3}]{import}-Alias.

\begin{lstlisting}[language={python3}, caption={\protect\lstset{language=python3} Das Turtle-Modul erhält in Zeile $1$ die Bezeichnung \lstinline{t}. Danach können wir die Funktionsaufrufe aus diesem Modul nur noch mit dieser Bezeichnung verwenden.}, label={lst:import-alias}]
import turtle as t

for _ in range(4):
	t.fd(100)
	t.lt(90)
t.done()
\end{lstlisting}
\end{example}

\begin{example}
\autoref{lst:import-alias-bsp} zeigt ein Programm mit zwei \lstinline[language={python3}]{import}-Anweisungen. Pro \lstinline[language={python3}]{import}-Anweisung wird ein \lstinline[language={python3}]{import}-Alias benutzt.

\begin{lstlisting}[language={python3}, caption={\protect\lstset{language=python3} Zwei \lstinline{import}-Anweisungen mit \lstinline{import}-Aliasing.}, label={lst:import-alias-bsp}]
import turtle as t
import random as r

a = r.randrange(25, 101)
for _ in range(4):
	t.fd(a)
	t.lt(90)
t.done()
\end{lstlisting}

\end{example}

Wir können für \textbf{jeden Modulimport} einen Alias benutzen. Dazu erweitern wir die \lstinline[language={python3}]{import}-Anweisung und verwenden am Ende das Schlüsselwort \lstinline[language={python3}]{as} mit einem gültigen Namen.

\begin{important}
Zwei \lstinline[language={python3}]{import}-Anweisung dürfen \textbf{nicht} den gleichen Alias besitzen.
\end{important}