% !TEX root = ../../../main.tex

\toggletrue{image}
\toggletrue{imagehover}
\chapterimage{wrong_times_table}
\chapterimagetitle{\uppercase{Wrong Times Table}}
\chapterimageurl{https://xkcd.com/2313/}
\chapterimagehover{Deep in some corner of my heart, I suspect that real times tables are wrong about 6x7=42 and 8x7=56.}

\chapter{Rechnen}
\label{ch:rechnen}

Mit einer Programmiersprache können wir natürlich auch Berechnungen durchführen. Das war schliesslich die ursprüngliche Motivation, Computer\footnote{\say{to compute} kann mit rechnen oder etwas berechnen übersetzt werden. Im deutschen Sprachraum ist auch die Bezeichnung Rechner für den Computer gebräuchlich.} zu bauen. Die Lernziele sind:\\

\lernziel{\autoref{ch:rechnen}, \nameref{ch:rechnen}}{
\begin{minipage}{\linewidth}
$\square$ \hspace{0.1cm} Sie erstellen Python-Programme, in denen mit Zahlen gerechnet wird.\\
$\square$ \hspace{0.1cm} Sie erklären, was ein arithmetischer Operator ist und geben ein Beispiel.\\
$\square$ \hspace{0.1cm} Sie erklären, was ein arithmetischer Ausdruck ist und geben ein Beispiel.
\end{minipage}
}

\vspace{-0.6cm}

\section{Addition, Subtraktion, Multiplikation und Division}
\label{sec:addition-subtraktion-multiplikation-division-und-potenzierung}

Wie bei einem Taschenrechner gibt es für jede Rechenart ein Zeichen: \textbf{Addition} \lstinline[language={python3}]{+}, \textbf{Subtraktion} \lstinline[language={python3}]{-}, \textbf{Multiplikation} \lstinline[language={python3}]{*} und \textbf{Division} \lstinline[language={python3}]{/}.

\begin{definition}[Arithmetischer Operator]
Der Fachbegriff für \say{Rechenzeichen} ist arithmetischer Operator und damit werden mathematische Berechnungen notiert.
\end{definition}

\vspace{-0.2cm}

Die Arithmetik ist ein Teilgebiet der Mathematik und beschreibt das Rechnen mit Zahlen. \autoref{lst:rechteck-1} zeigt ein Beispiel mit einer Multiplikation.

\begin{lstlisting}[language={python3}, label={lst:rechteck-1}, caption={Das Rechteck ist eineinhalbmal so lang wie breit.}]
import turtle as t
import random as r

a = r.randrange(100, 151)
b = 1.5 * a
for _ in range(2):
	t.fd(b)
	t.lt(90)
	t.fd(a)
	t.lt(90)
t.done()

\end{lstlisting}

\vspace{-0.6cm}

\cleancoderegel{\autoref{ch:rechnen}, \nameref{ch:rechnen}}{
\begin{cleancode}[Leerzeichen 5]
Links \textbf{und} rechts eines \textbf{arithmetischen Operators} fügen wir jeweils ein \textbf{Leerzeichen} ein.
\end{cleancode}
}

\autoref{lst:kreise-1} zeigt ein Beispiel mit einer Division und Subtraktion.

\begin{lstlisting}[language={python3}, label={lst:kreise-1}, caption={Die Anzahl der Kreise und der Radius des grösseren Kreises werden zufällig gewählt.}]
import random as r
import turtle as t

anzahl = r.randrange(4, 11)
drehwinkel = 360 / anzahl
radius_<@\color{black}{1}@> = r.randrange(30, 51)
radius_<@\color{black}{2}@> = radius_1 - 10
t.speed(0)
for _ in range(anzahl):
	t.circle(radius_<@\color{black}{1}@>)
	t.circle(radius_<@\color{black}{2}@>)
	t.lt(drehwinkel)
t.done()

\end{lstlisting}

\begin{important}
Einige Hinweise zu den Berechnungen:
\begin{itemize}
\item Wenn Sie \say{Kommazahlen} in Berechnungen verwenden wollen, müssen Sie für das Komma einen \textbf{Punkt} verwenden. 
\item Wenn Sie die \textbf{Reihenfolge von Berechnungen} beeinflussen wollen, können Sie wie in der Mathematik \textbf{runde Klammern} verwenden.
\item Im Gegensatz zur Mathematik müssen alle \textbf{arithmetischen Operatoren} immer angegeben werden. Ein fehlender arithmetischer Operator, wie im folgenden Code,
	
\begin{lstlisting}[language={python3}]
import random as r

x = r.randrange(1, 11)
ergebnis = 4x
\end{lstlisting}

führt zu einem \textbf{Fehler}.

\end{itemize}

\end{important}

Berechnungen sind Programmierbefehle, die in Python eine eigene Kategorie bilden.

\begin{definition}[Arithmetischer Ausdruck]
	Rechnungen mit \textbf{arithmetischen Operatoren} werden arithmetische Ausdrücke (engl. arithmetic expressions) genannt. Bei der Programmausführung werden arithmetische Ausdrücke stets von Python direkt \textbf{ausgewertet}. Dies bedeutet, Python ermittelt für den arithmetischen Ausdruck einen \textbf{Zahlenwert} (Integer oder Float).
\end{definition}

\begin{example}
Schauen wir uns \lstinline[language={python3}]{(163 * 3) - (77 * 4)} im Detail an:
$$
\begin{array}[t]{c}
\begin{array}[t]{cccccccccccc} 
\underbrace{\textrm{\texttt{(}}}_{\textrm{Klammer}} & \underbrace{\textrm{\texttt{163}}}_{\textrm{Wert}} & \underbrace{\textrm{\texttt{*}}}_{\textrm{arith. Op.}} & \underbrace{\textrm{\texttt{3}}}_{\textrm{Wert}} & \underbrace{\textrm{\texttt{)}}}_{\textrm{Klammer}} & \underbrace{\textrm{\texttt{-}}}_{\textrm{arith. Op.}} & \underbrace{\textrm{\texttt{(}}}_{\textrm{Klammer}} & \underbrace{\textrm{\texttt{77}}}_{\textrm{Wert}} & \underbrace{\textrm{\texttt{*}}}_{\textrm{arith. Op.}} & \underbrace{\textrm{\texttt{4}}}_{\textrm{Wert}} & \underbrace{\textrm{\texttt{)}}}_{\textrm{Klammer}}
\end{array} \\
\underbrace{\hspace{14cm}}_{\textrm{arithmetischer Ausdruck}}
\end{array}
$$
Mit arith. Op. ist der Begriff arithmetischer Operator gemeint.

\end{example}