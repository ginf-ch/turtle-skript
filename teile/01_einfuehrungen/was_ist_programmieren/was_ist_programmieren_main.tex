% !TEX root = ../../../main.tex

\toggletrue{image}
\toggletrue{imagehover}
\chapterimage{twitter_bot}
\chapterimagetitle{\uppercase{Twitter Bot}}
\chapterimageurl{https://xkcd.com/1646/}
\chapterimagehover{PYTHON FLAG ENABLE THREE LAWS.}


\chapter{Was ist Programmieren?}
\label{ch:was-ist-programmieren}

Was bedeutet Programmieren? \textbf{Programmieren} heisst, mit dem Computer zu kommunizieren.
Wir teilen dem Computer mit, was er zu tun hat und zwar in einer Sprache, die er versteht.
Sprachen, die der Computer versteht, nennen wir \textbf{Programmiersprachen}.
Wie jede natürliche Sprache besteht auch eine Programmiersprache aus Wörtern, die eine bestimmte Bedeutung haben.
Wir verwenden eine Programmiersprache, um dem Computer Anweisungen zu geben.
Deshalb nennen wir die Wörter einer Programmiersprache \textbf{Computerbefehle} oder kurz \textbf{Befehle}.
Manchmal wird auch das Wort \textbf{Instruktion} verwendet. Ein \textbf{Programm} besteht aus einer Reihe von Befehlen einer Programmiersprache.
Das Ziel des Programmierens ist es, eine Tätigkeit zu \textbf{automatisieren}.
Das bedeutet, dass wir die Ausübung einer Tätigkeit komplett dem Computer überlassen. \cite{einfach-informatik-programmieren}

\begin{definition}[Imperatives Programmieren]
    Beim imperativen Programmieren beschreibt das Programm, \textbf{wie} wir mit Computerbefehlen ein gewünschtes Ziel erreichen.
    Im Fokus steht die korrekte Reihenfolge der Befehle, um das Ziel schrittweise zu erreichen.
\end{definition}

In diesem Skript erlernen Sie die Grundlagen der imperativen Programmierung.
Wir verwenden dazu die Programmiersprache \textbf{Python} (\autoref{fig:python-logo}) in der Version 3.\footnote{Es gibt auch Python in der Version 2.}

\begin{figure}[htb]
    \centering
    \begin{minipage}[b][][b]{0.4\textwidth}
        \centering
        \includegraphics[scale=0.3]{python-logo}
        \caption{Das Python-Logo.\protect\footnotemark}
        \label{fig:python-logo}
    \end{minipage}
    \hfill
    \begin{minipage}[b][][b]{0.5\textwidth}
        \centering
        \includegraphics[scale=0.4]{turtle-star}
        \caption{Beispiel für eine Turtle-Grafik.\protect\footnotemark}
        \label{fig:turtle-graphic-beispiel}
    \end{minipage}
\end{figure}

\footnotetext{Bildquelle: \url{https://www.python.org/community/logos/}}
\footnotetext{Bildquelle: Screenshot des Programms.}

Um die Programmiergrundlagen zu erlernen, greifen wir auf das \textbf{Turtle-Modul} von Python zurück.
Mit diesem Modul bewegen wir eine Schildkröte (Turtle) mit einem \say{Stift} über den Bildschirm.
Wir können mit einem Python-Programm bestimmen, wie sich die Turtle bewegen soll.
\autoref{fig:turtle-graphic-beispiel} zeigt ein Beispiel einer Grafik, die mit der Turtle erstellt wurde.